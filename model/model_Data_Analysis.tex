\documentclass[UTF8,12pt,a4paper]{report}
%使用中文包
\usepackage{ctex}
%超链接所用的包
\usepackage[colorlinks,linkcolor=blue]{hyperref}
%代码
\usepackage{listings}
\usepackage{xcolor}
\lstset{
	columns=fixed,       
	numbers=left,                                        % 在左侧显示行号
	numberstyle=\tiny\color{gray},                       % 设定行号格式
	frame=none,                                          % 不显示背景边框
	backgroundcolor=\color[RGB]{245,245,244},            % 设定背景颜色
	keywordstyle=\color[RGB]{40,40,255},                 % 设定关键字颜色
	numberstyle=\footnotesize\color{darkgray},           
	commentstyle=\it\color[RGB]{0,96,96},                % 设置代码注释的格式
	stringstyle=\rmfamily\slshape\color[RGB]{128,0,0},   % 设置字符串格式
	showstringspaces=false,                              % 不显示字符串中的空格
	language=c++,                                        % 设置语言
}
% 统一修改正文和数学字体为Adobe Utopia
\usepackage{fourier}
\usepackage{indentfirst}
\usepackage{graphicx} %插入图片的宏包
\usepackage{float} %设置图片浮动位置的宏包
\usepackage{subfigure} %插入多图时用子图显示的宏包
\setlength{\parindent}{2em}
\title{Big Data Analysis Experiment Report }
\author{Zhiyang Feng\\201900130012}
\date{\today}
\begin{document}
	\maketitle
	\newpage
	This is the model of Experiment Report.And you can edit this file to any style that you want.
	
	
	这是实验报告模板,你可以将其编辑成任何你想要的风格。
	
	The methods of inserting pictures are in \href{https://zhuanlan.zhihu.com/p/32925549}{图片插入方法}.So if you want to know some knowledge about this,you need to read it.
	
	关于插入图片的方法到\href{https://zhuanlan.zhihu.com/p/32925549}{图片插入方法}。如果有需要请自行阅读。
	
	
	The methods of inserting tables are in \href{https://blog.csdn.net/xovee/article/details/109254872}{表格插入方法}.So if you want to know some knowledge about this,you need to read it.
	
	关于插入表格的部分方法到\href{https://blog.csdn.net/xovee/article/details/109254872}{表格插入方法}。如果有需要请自行阅读。
	\section{Experiment Requirement}
	
	This part is the requirements of experiment.
	
	
	这个部分书写实验要求
	
	\begin{enumerate}
		\item[1.] The first requirement.
		\item[2.] The second requirement.
		\item[3.] The third requirement.
	\end{enumerate}


	\section{Experiment Environment}
	
	This part is the requests of this Experiment.
	
	
	这部分书写实验要求的环境。
	
	\textbf{HUAWEI MATEBOOK 14,Jupyter Notebook,Python 3.8}
	
	\section{Experiment Ideas}
	This part is the ideas of this experiment.
	
	
	这部分书写实验的思路。
	
	\begin{enumerate}
	\item[1.] The first idea.
	\item[2.] The second idea.
	\item[3.] The third idea.
	\end{enumerate}
	
	\section{Experiment Stages}
	This part is the stages of this experiment.
	
	
	这部分书写实验的步骤。
	
	\begin{enumerate}
	\item[1.] The first stage.
	\item[2.] The second stage.
	\item[3.] The third stage.
	\end{enumerate}
	
	
	\section{Experiment Result \& Thinking}
	This part is the Result \& thinkings of this experiment.
	
	
	这部分书写实验的结果和思考。
	
	
	\section{Summary}
	This part is the Summary of this experiment.
	
	
	这部分书写实验的总结
	
	
	\section{Appendix}。
	
	
	This part is the code of this experiment.
	
	
	这个部分中包含本次实验过程中的代码。
	\begin{lstlisting}
		% 代码段
		#include<iostream>
		using namespace std;
		int main()
		{
			cout<<"Hello World!"<<"\n";
			return 0;	
		}
	\end{lstlisting}
	
\end{document}
